\documentclass[useAMS,usenatbib]{mnras}
\usepackage{color}
\usepackage{graphicx}
\usepackage{dcolumn}
\usepackage{bm}
\usepackage{amssymb}
\usepackage{latexsym}
\usepackage[T1]{fontenc}
\usepackage{aecompl} 

\newcommand{\hMsun}{{\ifmmode{h^{-1}{\rm
        {M_{\odot}}}}\else{$h^{-1}{\rm{M_{\odot}}}$~}\fi}} 
\newcommand{\hMpc}{{\ifmmode{h^{-1}{\rm Mpc}}\else{$h^{-1}$Mpc }\fi}}

%%%%%%%%%%%%%%%%%%%%%%%%%%%%%%%%%%%%%%%%%%%%%%%%

\begin{document}

\title[Constraining CPL via clustering shells of BOSS]{Cosmological constraints via clustering shells: 
constraining the Chevallier-Polarski-Linder parametrization using the SDSS-III BOSS data}

\author[Xiao-Dong~Li, Yuting Wang, Gong-bo Zhao, Changbom Park, Hyunbae Park, Cristiano G. Sabiu]
{ Xiao-Dong Li$^{1,2,\dagger}$,  Yuting Wang$^{2}$, Gong-bo Zhao$^{2}$, Changbom Park$^{1}$, Hyunbae Park$^{2}$, 
Cristiano G. Sabiu$^{2,\star}$, Juhan Kim$^{3,1}$\\
$^1$School of Physics, Korea Institute for Advanced Study, 85 Heogi-ro, Dongdaemun-gu, Seoul 130-722, Korea\\
$^2$Korea Astronomy and Space Science Institute, 776, Daedeokdae-ro, Yuseong-gu, Daejeon, 305-348, Korea\\
$^3$Center for Advanced Computation, Korea Institute for Advanced Study, 85 Hoegi-ro, Dongdaemun-gu, Seoul 130-722, Korea\\
$^{\dagger}$xiaodongli@kias.re.kr\\
$\star$Corresponding Author: csabiu@kasi.re.kr}




%\date{Accepted 1988 December 15. Received 1988 December 14; in original form 1988 October 11}

% \begin{keywords}
% methods: data analysis -- methods: statistical -- Galaxies:
% kinematics and dynamics -- Cosmology: observations -- large-scale
% structure of universe
% \end{keywords}

\pagerange{\pageref{firstpage}--\pageref{lastpage}} \pubyear{2002}

\maketitle

\label{firstpage}

\begin{abstract}
Li et al. 2016 proposed to use the redshift dependence of the Alcock-Paczynski to constrain cosmological parameters.
Tight constraints on $\Omega_m$ and $w$ were derived using the SDSS-III BOSS galaxies.
In this paper we extend the analysis to put constraints on the Chevallier-Polarski-Linder parametrization.
To explore the high dimension parameter space, we measure the 2-point correlation function in some fiducial cosmology,
and map out its value in the other cosmologies.
%The accuracy is below 1\%.
We derive constraints of ...
\end{abstract}

% \begin{keywords}
% circumstellar matter -- infrared: stars.
% \end{keywords}

\section{Introduction}


DE is very important. 
\citep{Li2016} proposed to use redshift dependence of AP to constrain cosmological parameters.
Very good results obtained from SDSS-III BOSS galaxies.

\citep{Li2016} only considered constraints on $\Omega_m$ and $w$ in a flat Universe.
In this paper we extend to CPL parameters.




\section{Methodology}

The 2-point correlation function (2pCF) is measured in a fiducial cosmology,
and converted to obtain the measurement in other cosmologies using the relation of 
\begin{eqnarray}
 s_{\rm t} = s_{\rm f} \sqrt{\alpha_{\parallel}^2 \mu_{\rm f}+\alpha_{\bot}(1-\mu_{\rm f}^2)}, \\
 \mu_{\rm t} = \mu_{\rm f} \frac{\alpha_\parallel}
 {\sqrt{\alpha_{\parallel}^2 \mu_{\rm f} +\alpha_{\bot}(1-\mu_{\rm f}^2)}}
\end{eqnarray}
Let us call this ``approximate 2pCF'' method.

We use a number of $n_{\mu}$=20, 21, ... 25 bins for the value of $\xi_{\Delta s}(\mu)$ in the $\mu$ space.
For the removal of fiber collision and the strong FoG near the LOS cut we take a cut $\mu_{\rm max} = 0.97$.
The result is insensitive to the choices of $n_\mu$ and $\mu_{\rm max}$.




\section{Input-Output Test}


\section{Reproducing wCDM}

We firstly try to reproduce the constraints on $\Omega_m$ and $w$ in \citep{Li2016}:
\begin{equation}\label{eq:wcdm_constrain_default}
 \Omega_m=0.301\pm 0.006, w=−1.054\pm 0.025.
\end{equation}
Using the technique of approximate 2pCF and taking a fiducial of $\Omega_m=0.26, w=-1$, we obtains
\begin{equation}
\Omega_m = 0.301 \pm 0.006,\ w=-1.053\pm 0.034,
%    1  0.3010870E+00  0.7818562E-02  0.2973018E+00  0.3045011E+00  0.2883318E+00  0.3150334E+00   \Omega_m
%    2  0.6886453E+00  0.8507156E-02  0.6802557E+00  0.6975813E+00  0.6708444E+00  0.7047652E+00   h
%    3 -0.1053343E+01  0.3395215E-01 -0.1087474E+01 -0.1020395E+01 -0.1121890E+01 -0.9819443E+00   w
\end{equation}
while taking a different fiducial cosmology yields
\begin{eqnarray}
\Omega_m = 0.300 \pm 0.008,\ w=-1.057\pm 0.034\ (\Omega_m=0.26, w=-0.6),\\    
%param  mean           sddev          lower1         upper1         lower2         upper2         
%    1  0.3000053E+00  0.7703564E-02  0.2956289E+00  0.3032603E+00  0.2886989E+00  0.3142216E+00   \Omega_m
%   2  0.6898875E+00  0.8383848E-02  0.6805430E+00  0.6979808E+00  0.6721509E+00  0.7035974E+00   h
%    3 -0.1057455E+01  0.3363486E-01 -0.1090917E+01 -0.1020533E+01 -0.1116405E+01 -0.9873376E+00   w
\Omega_m = 0.304 \pm 0.007,\ w=-1.040\pm 0.030\ (\Omega_m=0.26, w=-1.4),\\
%    1  0.3037843E+00  0.7243439E-02  0.3005621E+00  0.3073846E+00  0.2911318E+00  0.3154794E+00   \Omega_m
%    2  0.6852584E+00  0.7600452E-02  0.6780053E+00  0.6935965E+00  0.6714114E+00  0.7010620E+00   h
%    3 -0.1040019E+01  0.3026509E-01 -0.1069579E+01 -0.1011192E+01 -0.1105126E+01 -0.9823171E+00   w
\Omega_m = 0.301 \pm 0.009,\ w=-1.055\pm 0.039\ (\Omega_m=0.31, w=-1.0),\\
%1  0.3007960E+00  0.8740571E-02  0.2960415E+00  0.3046011E+00  0.2873819E+00  0.3166509E+00   \Omega_m
    %2  0.6890261E+00  0.9820733E-02  0.6785434E+00  0.6990516E+00  0.6684322E+00  0.7058833E+00   h
    %3 -0.1054663E+01  0.3932623E-01 -0.1094898E+01 -0.1014076E+01 -0.1125992E+01 -0.9715273E+00   w
\Omega_m = 0.302 \pm 0.008,\ w=-1.050\pm 0.035\ (\Omega_m=0.31, w=-0.6),\\    
%    1  0.3015603E+00  0.8075695E-02  0.2970461E+00  0.3054528E+00  0.2890849E+00  0.3157329E+00   \Omega_m
%    2  0.6879760E+00  0.8903611E-02  0.6783486E+00  0.6972412E+00  0.6704429E+00  0.7035974E+00   h
%    3 -0.1050127E+01  0.3579852E-01 -0.1087200E+01 -0.1012397E+01 -0.1116100E+01 -0.9792085E+00   w
\Omega_m = 0.301 \pm 0.009,\ w=-1.052\pm 0.038\ (\Omega_m=0.31, w=-1.4).\\    
%    1  0.3010573E+00  0.8626961E-02  0.2960266E+00  0.3049617E+00  0.2883365E+00  0.3169329E+00   \Omega_m
%    2  0.6886086E+00  0.9597560E-02  0.6785247E+00  0.6981217E+00  0.6679604E+00  0.7046714E+00   h
%    3 -0.1052455E+01  0.3836365E-01 -0.1091669E+01 -0.1013558E+01 -0.1123257E+01 -0.9702918E+00   w
\end{eqnarray}
so the derived constraints are rather insensitive to the fiducial cosmology.

We will use two fiducial cosmologies,
the $\Omega_m=0.26,\ 0.31$ $\Lambda$CDM models which are the best-fit cosmologies of WMAP5 \cite{WMAP5} and \cite{Planck}.
Averaging the 2pCFs inferred from the two fiducial cosmologies we get
\begin{equation}
\Omega_m = 0.301 \pm 0.008,\ w=-1.054\pm 0.036.
%    1  0.2991453E+00  0.7662944E-02  0.2951697E+00  0.3023594E+00  0.2871742E+00  0.3130522E+00   \Omega_m
%    2  0.6910298E+00  0.8322516E-02  0.6830850E+00  0.6999792E+00  0.6728376E+00  0.7055694E+00   h
%    3 -0.1062378E+01  0.3310274E-01 -0.1097275E+01 -0.1030780E+01 -0.1125694E+01 -0.9921285E+00   w
\end{equation}
The constraints are well consistent with Equation \ref{eq:wcdm_constrain_default},
except a slightly larger error bar due to the limitation of approximation method
\footnote{Equation \ref{eq:wcdm_constrain_default} was derived using $n_{\mu}$=25-35.
For constraints from the approximate 2pCF we use $n_{\mu}$=20-25.
We could get a tighter constraint with similar means values of $\Omega_m$ and $w$ by using larger $n_\mu$.
However, due to the limited accuray of the approximation, the contour becomes noisy if $n_\mu$ is larger.
So we use $n_\mu<25$ for a consideration of safty. }.
%from the 
%The approximate 2pCF may not be accurate enough }


%Using other fiducial cosmologies we derive similar cosmological constraints...
%We simply adopt the $\Omega_m=0.26, w=-1$ as the fiducial cosmology in the following analysis...

\section{Constraints on CPL}

Combining our method with CMB+BAO+JLA+$H_0$ we derive the following constraints on CPL parameters
\begin{equation}
\Omega_m = 0.300 \pm 0.076, w_0 = -1.056 \pm 0.061, w_a = -0.04 \pm 0.27,
    %1  0.2999656E+00  0.7616334E-02  0.2958745E+00  0.3035033E+00  0.2878169E+00  0.3132870E+00   \Omega_m
    %2  0.6904801E+00  0.8675444E-02  0.6808558E+00  0.6993898E+00  0.6722945E+00  0.7052207E+00   h
    %3 -0.1056468E+01  0.6096323E-01 -0.1116874E+01 -0.9965903E+00 -0.1175041E+01 -0.9287388E+00   w
    %4 -0.3808376E-01  0.2716607E+00 -0.3105235E+00  0.2321461E+00 -0.6033103E+00  0.4782076E+00   w_a    
\end{equation}
whose error bars are $~30-40\%$ smaller than the result without adding our method
\begin{equation}
\Omega_m = 0.309 \pm 0.096, w_0 = -0.938 \pm 0.109, w_a = -0.38 \pm 0.41.
%    1  0.3086006E+00  0.9639142E-02  0.3039045E+00  0.3131096E+00  0.2926126E+00  0.3249674E+00   \Omega_m
%    2  0.6809952E+00  0.1054577E-01  0.6703974E+00  0.6916269E+00  0.6605231E+00  0.7026640E+00   h
%    3 -0.9377990E+00  0.1092489E+00 -0.1046659E+01 -0.8303874E+00 -0.1149980E+01 -0.7142246E+00   w
%    4 -0.3777249E+00  0.4069512E+00 -0.7694896E+00  0.1941066E-01 -0.1285417E+01  0.3473089E+00   w_a
\end{equation}

The result is fully consistent with a dark energy component having no evolution.




%\subsection{Some Plottings, Discussions, Definition of $\chi^2$, ...}

%{\bf Figure: $\xi$ curves, in different cosmologies, without and with RSD}

\begin{figure*}
   \centering{
%   \includegraphics[height=8cm]{Tpcf--plot--UnNorm.eps}
%   \includegraphics[height=8cm]{Tpcf--plot--Normed.eps}
%    \includegraphics[height=8cm]{smu.eps}
   }
   \caption{\label{fig_TpCF}
   Haha
   }
\end{figure*}




\section*{Acknowledgments}

We thank the Korea Institute for Advanced Study for providing computing resources (KIAS Center for Advanced Computation Linux Cluster System).
We thank Seokcheon Lee and Graziano Rossi for many helpful discussions.


\begin{thebibliography}{}

\bibitem[Alcock \& Paczynski(1979)]{AP1979}
Alcock, C., \& Paczynski, B. 1979, Nature, 281, 358  

\bibitem[Anderson et al.(2013)]{Anderson2013}
Anderson, L., Aubourg, \'E., Bailey, S. et al. 2014, MNRAS, 441, 24  


\bibitem[Bassett et al.(2002)]{Bassett2002}
Bassett, B.A., Kunz, M., Silk, J., \& Ungarelli, C. 2002, MNRAS, 336, 1217

\bibitem[Ballinger, Peacock \& Heavens 1996]{Ballinger1996}
Ballinger, W.E., Peacock, J.A., \& Heavens, A.F. 1996, MNRAS, 282, 877  


\bibitem[Beutler et al.(2013)]{Beutler2013}
Beutler, F., Saito, S., Seo, H.-J., et al. 2013, MNRAS, 443, 1065

\bibitem[Blake et al.(2011)]{Blake2011}
Blake, C., Glazebrook, K., Davis, T. M., 2011, MNRAS, 418, 1725  


\bibitem[Bueno Belloso et al. (2012)]{BB2012}
Bueno Belloso, A., Pettinari, G.W., Meures, N., \& Percival, W.J. 2012, Phys. Rev. D, 86, 023530

\bibitem[Chevallier \& Polarski(2001)]{CP2001}
Chevallier, M., Polarski, D. 2001, Int. J. Mod. Phys. D, 10, 213


\bibitem[Choi et al.(2010)]{choi 2010}
Choi, Y.-Y., Park, C., Kim, J., Gott, J.R., 
Weinberg, D.H., Vogeley, M.S., \& Kim, S.S. 2010, ApJS, 190, 181  

%\bibitem[Chuang et al.(2013)]{Chuang2013}
%Chuang, C.-H., Prada, F., Beutler, F., et al. 2013, arXiv:1312.4889  

\bibitem[Chuang \& Wang(2012)]{ChuangWang2012}
Chuang, C.-H., \& Wang, Y. 2012, MNRAS, 426, 226  

\bibitem[Corasaniti \& Copeland(2003)]{Corasaniti2003}
Corasaniti, P.S., Copeland, E.J. 2003, Phys. Rev. D, 67, 063521

%\bibitem[Gingold \& Monaghan(1977)]{GM1977}
%Gingold, R.A., \& Monaghan, J.J. 1977, MNRAS, 181, 375  

\bibitem[Gott et al.(2009)]{gott 2009}
Gott, J.R., Choi, Y.-Y., Park, C., \& Kim, J. 2009, ApJ, 695, L45  

\bibitem[Gott et al.(2008)]{gott 2008}
Gott, J.R., Hambrick, D.C., Vogeley, M.S., Kim, J., Park, C., Choi, Y.-Y.,
Cen, R., Ostriker, J.P., \& Nagamine, K. 2008, ApJ, 675, 16  

\bibitem[Jackson (1972)]{FoG}
Jackson J., 1972, MNRAS, 156, 1

\bibitem[Jennings et al.(2011)]{Jennings2011}
Jennings, E., Baugh, C.M., \& Pascoli, S. 2011, MNRAS, 420, 1079  

\bibitem[Jeong et al.(2014)]{Jeong2014}
Jeong, D., Dai, L., Kamionkowski, M., \& Szalay, A.S. 2014, arXiv:1408.4648

\bibitem[Kim \& Park(2006)]{kim and park 2006}
Kim, J., \& Park, C. 2006, ApJ, 639, 600  

\bibitem[Kim et al.(2009)]{2009ApJ...701.1547K} Kim, J., Park, C., Gott, 
J.~R., III, \& Dubinski, J.\ 2009, ApJ, 701, 1547 

\bibitem[Kim et al.(2011)]{horizonrun}
Kim, J., Park, C., Rossi, G., Lee, S.M., \& Gott, J.R., 2011, JKAS, 44, 217  

\bibitem[Komatsu et al.(2011)]{komatsu 2011}
Komatsu, E., Smith, K. M., Dunkley, J., et al. 2011, ApJS, 192, 18  

\bibitem[Landy \& Szalay(1993)]{1993ApJ...412...64L} S.~D. Landy, \& A.~S. Szalay, \ 1993, ApJ, 412, 64 

\bibitem[Lavaux \& Wandelt(2012)]{LavausWandelt1995}
Lavaux, G., \& Wandelt, B.D. 2012, ApJ, 754, 109  

\bibitem[Levi et al.(2013)]{2013arXiv1308.0847L} 
Levi, M., Bebek, C., Beers, T., et al.\ 2013, arXiv:1308.0847 

\bibitem[Li et al.(2011)]{Li2011}
Li, M., Li, X.-D., Wang, S., \& Wang, Y. 2011, Commun. Theor. Phys., 56, 525

\bibitem[Li et al.(2014)]{Li2014}
Li, X.-D., Park, C., Forero-Romero, J., \& Kim, J. 2014, ApJ, 796, 137

\bibitem[Li et al.(2015)]{Li2015}
Li, X.-D., Park, C., Sabiu, C.G., \& Kim, J. 2015, MNRAS, 450, 807 

\bibitem[Li et al.(2016)]{Li2016}
Li, X.-D., Park, C., Sabiu, C.G., \& Kim, J. 2016, submitted to ApJ


\bibitem[Linder(2003)]{Linder2003}
Linder, E.V. 2003, Phys. Rev. Lett., 90, 091301

\bibitem[Linder et al.(2014)]{Linder2013}
Linder, E.V., Minji, O., Okumura, T., Sabiu, C.G., \& Song, Y.-S. 2014, Phys. Rev. D, 89, 063525  

\bibitem[L{\'o}pez-Corredoira(2014)]{2014ApJ...781...96L} 
L{\'o}pez-Corredoira, M.\ 2014, ApJ, 781, 96 

\bibitem[Marinoni \& Buzzi(2010)]{Marinoni2010}
Marinoni, C., \& Buzzi, A. 2010, Nature, 468, 539  

\bibitem[Matsubara \& Suto(1996)]{Matsubara1996}
Matsubara T., \& Suto, Y. 1996, ApJ, 470, L1  

\bibitem[Outram et al.(2004)]{Outram2004}
Outram, P.J., Shanks, T., Boyle, B.J., Croom, S.M., Hoyle, F., Loaring, N.S., 
Miller, L., \& Smith, R.J. 2004, MNRAS, 348, 745  

%\bibitem[Parejko et al.(2013)]{Parejko2013}
%Parejko, J. K., Sunayama, T., Padmanabhan, N., et al. 2013, MNRAS, 429, 98  

\bibitem[Park et al.(2005)]{park 2005}
Park, C., Kim, J., \& Gott, J.R. 2005, ApJ, 633, 1  

\bibitem[Park \& Kim(2010)]{topology}
Park, C., \& Kim, Y.-R. 2010, ApJL, 715, L185  

\bibitem[Press \& Shechter(1974)]{PS1974}
Press, W.H., \& Schechter, P.L. 1974, ApJ, 187, 425

\bibitem[Perlmutter et al.(1999)]{Perl1999}
Perlmutter, S., Aldering, G., Goldhaber, G., et al. 1999, ApJ, 517, 565  

\bibitem[Reid et al.(2012)]{Reid2012}
Reid, B. A., Samushia, L., White, M., et al. 2012, MNRAS, 426, 2719  

\bibitem[Riess et al.(1998)]{Riess1998}
Riess, A. G., Filippenko, A. V., Challis, P., et al. 1998, AJ, 116, 1009  

 %\bibitem[Ross et al.(2012)]{2012MNRAS.424..564R} A.~J. Ross, W.~J. Percival, 
%A.~G. S{\'a}nchez, et al.\ 2012, MNRAS, 424, 564 

\bibitem[Ryden(1995)]{Ryden1995}
Ryden, B.S. 1995, ApJ, 452, 25  

%\bibitem[Samushia et al.(2014)]{Samushia2014}
%Samushia, L., Reid, B. A., White, M., et al. 2014, MNRAS, 439, 3504  

\bibitem[Sanchez et al.(2013)]{Sanchez2013}
Sanchez, A. G., Kazin, E. A., Beutler, F., et al. 2013, MNRAS, 433, 1202  

\bibitem[Sutter et al.(2014)]{Sutter2014}
Sutter, P.M., Pisani, A., Wandelt, B.D., \& Weinberg, D.H. 2014, MNRAS, 443, 2983

\bibitem[Song et al.(2014)]{2014arXiv1407.2257S} Song, Y.-S., Sabiu, C.~G., 
Okumura, T., Oh, M., \& Linder, E.~V.\ 2014, JCAP, 12, 005 

%\bibitem[Tojeiro \& Percival(2011)]{Tojeiro2011}
%Tojeiro R., \& Percivial W.J. 2011, MNRAS, 417, 1114  

%\bibitem[Tojeiro et al.(2012)]{Tojeiro2012}
%Tojeiro, R., Percival, W. J., Wake, D. A., et al. 2012, MNRAS, 424, 136 

\bibitem[Viana \& Liddle(1996)]{VL1996}
Viana, P.T.P., \& Liddle, A.R. 1996, MNRAS, 281, 323


\end{thebibliography}

\bsp

\label{lastpage}

\end{document}


